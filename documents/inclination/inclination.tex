\documentclass[12pt, preprint]{aastex}
\usepackage{bm}
\input{vc}

\newcommand{\setof}[1]{\left\{{#1}\right\}}
\newcommand{\given}{\,|\,}
\newcommand{\dd}{\mathrm{d}}
\newcommand{\catalog}{\bm{Q}}
\newcommand{\pars}{\bm{\theta}}

\begin{document}

\title{Removing inclination noise}
\author{some combination of DWH, DFM, and others}
\date{NOT READY / \texttt{\githash}}

\begin{abstract}
Many observations of kinematic systems (such as galactic black holes,
stellar rotation, and exoplanets) suffer from an inclination
uncertainty.
When the system is unresolved the observable contains a sine of the
inclination ($\sin i$).
Here we show that it is possible to infer the true distribution of
rotation velocities $v$ from noisy, censored measurements of $v\,\sin
i$ (or the true distribution of masses $M$ from measurements of
$M\,\sin i$).
This is a classic problem in ``non-blind deconvolution''.
We frame the problem as a hierarchical probabilistic inference; in
this formulation, the whole distribution of the observable $v\,\sin i$
is compared to that expected from a distribution of latent true
values $v$ and the expected inclination distribution and other noise
sources and censoring.
This permits us to write a likelihood function for the catalog of
observed values, given a parameterized or non-parametric model for the
distribution of latent values.
We show that for realistic simulated rotation velocity data, the
method returns reliable inferences for the distribution of true
rotation velocities, even when the velocity distribution model used in
the inference does not include the model with which the velocities
were originally drawn.
One interesting consequence or application of the probabilistic model
is that it provides informative posterior information, for each object
in the catalog, about its true inclination.
\end{abstract}

\section{Introduction}

...For specificity we will consider stellar rotation.

\section{Method}

...Equation dump...
\begin{eqnarray}
Q &\equiv& v\,\sin i
\\
f(Q\given\pars) &=& \int P(Q)\,p(Q\given v,i)\,f(v\given\pars)\,p(i)\,\dd i\,\dd v
\\
\int p(x)\,\dd x &=& 1
\\
\int f(x)\,\dd x &=& E[N]
\\
\ln p(\catalog\given\pars) &=& \sum_{n=1}^N f(Q_n\given\pars) - \ln \int f(Q\given\pars)\,\dd Q
\\
\catalog &\equiv& \setof{Q_n}_{n=1}^N
\\
p(\pars\given\catalog) &=& \frac{1}{Z}\,p(\catalog\given\pars)\,p(\pars)
\\
p(i_n\given\catalog) &=& \int p(i_n\given\pars)\,p(\pars\given\catalog)\,\dd\pars
\\
p(i_n\given\pars) &=& \frac{1}{Z_n}\,\int p(Q_n\given v_n,i_n)\,f(v_n\given\pars)\,p(i_n)\,\dd v_n
\\
p(v_n\given\catalog) &=& \int p(v_n\given\pars)\,p(\pars\given\catalog)\,\dd\pars
\\
p(v_n\given\pars) &=& \frac{1}{Z_n}\,\int p(Q_n\given v_n,i_n)\,f(v_n\given\pars)\,p(i_n)\,\dd i_n
\end{eqnarray}

\section{Experiments}

...To make an experiment, we need to generate fake data from some true
distribution, and add some noise.
We \emph{also} have to choose a form for the latent distribution
$f(Q\given pars)$; that is, a parameterization $\theta$
Once we have chosen both, we have an experiment.

\section{Discussion}

...Method generalizes to more complex situations; for example when the
censoring depends on variables beyond $Q$, or when the direct
observable is not precisely $Q$ but something intermediate.

\acknowledgements
It is a pleasure to thank...
Funding...
Code...

\end{document}
