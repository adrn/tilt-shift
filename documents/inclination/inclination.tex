\documentclass[12pt, preprint]{aastex}

\begin{document}

\title{Removing inclination noise}
\author{some combination of DWH, DFM, and others}
\date{}

\begin{abstract}
Many observations of dynamical systems---galactic black holes, stellar
rotation, and exoplanets, for examples---suffer from an inclination
uncertainty.
When the system is unresolved the observable contains a sine ($\sin
i$) of the inclination ($i$).
Here we show that it is possible to infer the true distribution of
rotation velocities $v$ from noisy, censored measurements of $v\,\sin
i$ (or the true distribution of masses $M$ from measurements of
$M\,\sin i$).
This is a classic problem in non-blind ``deconvolution''.
We frame the problem as a hierarchical probabilistic inference; in
this formulation, the whole distribution of the observable $v\,\sin i$
is compared to that expected from a distribution of latent ``true''
values $v$ and the expected inclination distribution and other noise
sources and censoring.
This permits us to write a likelihood function for the catalog of
observed values, given a parameterized or non-parametric model for the
distribution of latent values.
We show that for realistic simulated rotation velocity data, the
method returns reliable inferences for the distribution of true
rotation velocities, even when the velocity distribution model used in
the inference does not include the model with which the velocities
were originally drawn.
One interesting consequence or application of the probabilistic model
is that it provides informative posterior information, for each object
in the catalog, about its true inclination.
\end{abstract}

\section{Introduction}

\section{Method}

\section{Experiments}

\section{Discussion}

\acknowledgements
It is a pleasure to thank...
Funding...
Code...

\end{document}
